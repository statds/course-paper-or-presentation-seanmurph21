\documentclass[12pt]{article}

%% preamble: Keep it clean; only include those you need
\usepackage{amsmath}
\usepackage[margin = 1in]{geometry}
\usepackage{graphicx}
\usepackage{booktabs}
\usepackage{natbib}

% highlighting hyper links
\usepackage[colorlinks=true, citecolor=blue]{hyperref}

\title{Comparison of Multi-Class Classification Methods}
\author{Sean Murphy\\
  Statistics Major\\
  University of Connecticut
}

\begin{document}
\maketitle

\begin{abstract}
THE ABSTRACT WILL BE WRITTEN AFTER THE REST OF THE PAPER HAS BEEN WRITTEN. 
\end{abstract}


\section{Introduction}
\label{sec:intro}

% Use this section to answer three questions:
% Why is the topic important/interesting?
% What has been done on this topic in the literature?
% What is your contribution?

%To cite a reference, here are examples.
% \citet{xie2015dynamic} did something ... 

% A lot of work has been done \citep[e.g.,][]{xie2015dynamic}.
% Some parametric bootstrap sample size approach was proposed by
% \citet{dwivedi2017analysis}. 


The rest of the paper will be organized in the following way.  
An introduction to the red wine data set will be presented in 
Section~\ref{sec:data}.  This section will describe the variables 
and observations contained in the dataset, and will include a brief 
overview of summary statistics.  Next, a methodological overview of 
this analysis will be given in Section~\ref{sec:meth}.  Each of the 
classification methods and their underlying assumptions will be 
presented in Section~\ref{sec:class}.  The classification metrics 
which will be used to compare the predictive effectiveness of each 
model will be introduced in Section~\ref{sec:metr}.  After this, 
the results will be presented with tables and figures in 
Section~\ref{sec:resu}.  Finally, once the results of the analysis 
have been described, a discussion of their implications and suggestions 
for further study will be covered in Section~\ref{sec:disc}.


\section{Data}
\label{sec:data}

The data that will be analyzed in this study comes from the UC Irvine 
Machine Learning Repository.  It is a dataset from 2009 containing 
information about a sample of red "Vinho Verde" wine from north Portugal.  
The dataset contains $n=1,599$ observations (different wine samples) of 
twelve variables, eleven of which are continuous numeric variables.  
These continuous variables are different physiochemical measurements of 
the wine samples: fixed acidity, volatile acidity, citric acid, residual 
sugar, chlorides, free sulfur dioxide, total sulfur dioxide, density, 
pH, sulfates and alcohol.  The other variable contained in the dataset 
is wine quality, which takes integer values from 0-10 (0 being lowest 
quality, 10 being highest quality).  Though numeric, this variable will 
be treated as having 11 classes, and will be the response variable in 
the analysis.  The goal is to predict the class of wine quality for a 
given sample of red wine based on its physiochemical properties.  

PROVIDE HERE SOME SUMMARY STATISTICS OF THE DATASET

% Use this section to describe the data that helps to answer your research
% questions. Recall Einstein's equation
% \begin{equation}
  % \label{eq:mc2}
  % E = m c^2,
% \end{equation}
% which states that the energy $E$ of a particle in its rest frame as the product
% of mass ($m$) with the speed of light squared ($c^2$).

\section{Methods}
\label{sec:meth}

OVERVIEW PARAGRAPH

\subsection{Classification Methods}
\label{sec:class}

In classification, multinomial regression is the name given to the 
logistic regression model that predicts response variable values 
for $K > 2$ classes.  Since it is an extension of logistic regression 
for binary classification, it shares many of the same characteristics.  
Logistic regression models in general are modifications to linear 
regression models that make them fit to return values between 0 and 1.  
Recall that a multiple linear regression model takes the following form:
\begin{equation}
  \label{eq:linreg}
  Y = \beta_0 + \beta_1X_1 + ... + \beta_pX_p + \epsilon.
\end{equation}
In Equation~\eqref{eq:linreg}, we see that there is nothing restricting 
$Y$ from taking values $> 1$ or $< 0$.  This is problematic if we wish 
for our model to predict the probability that the response variable 
falls within a particular class, since probabilities can only take values 
between 0 and 1.  In order to solve this problem, the binary classification 
logistic regression model takes the following form:
\begin{equation}
  \label{eq:logreg}
  p(X) = 
  \frac{e ^ {\beta_0 + \beta_1X_1 + ... + \beta_pX_p}} 
  {1 + e ^ {\beta_0 + \beta_1X_1 + ... + \beta_pX_p}}.
\end{equation}
In Equation~\eqref{eq:logreg}, we see that this problem is essentially 
solved by exponentiation.  The multiple linear regression model is placed 
into the exponent of $e$ to ensure that the numerator and denominator will 
always be positive.  This means that the overall model will not return values 
$< 0$.  Additionally, because the numerator will never exceed the denominator, 
the fraction will always be $\leq 1$.  Thus, the model outputs values in the 
desired range for predicting probabilities.  Since in binary classification 
there are only two classes, the output of this model represents the probability 
that $Y$ will fall into one of the predetermined classes.  The probability that 
$Y$ will fall into the other class is simply $1 - p(X)$.

 In multinomial logistic regression, however, there are more than 2 classes 
 into which the response variable could fall.  The binary model is generalized 
 to $K > 2$ classes in the following manner:
 \begin{equation}
  \label{eq:multireg}
  Pr( Y = k | X = x ) = 
  \frac{e ^ {\beta_k0 + \beta_k1X_1 + ... + \beta_kpX_p}}
  { \sum_{l = 1} ^ {K}  e ^ {\beta_l0 + \beta_l1X_1 + ... + \beta_lpX_p}},
\end{equation}
where $x$ is a vector of the $p$ predictor values.  In 
Equation~\eqref{eq:multireg}, we see that the numerator is the exponentiated 
linear model for class $k$, while the denominator is the sum of the exponentiated 
linear models for all $K = k$ classes.  It should be noted that this is not the 
only way of generalizing the binary logistic regression model to $K >2$ classes.  
The model presented in Equation~\eqref{eq:multireg} is known as the \textit{softmax} 
coding \citep{james2021introduction}.  There are other versions of multinomial 
regression models, but we will not use them and thus not elaborate on them further 
for this analysis.  

LINEAR DISCRIMINANT ANALYSIS

QUADRATIC DISCRIMINANT ANALYSIS 

K NEAREST NEIGHBORS

NAIVE BAYES

SUPPORT VECTOR MACHINES

\subsection{Classification Metrics}
\label{sec:metr}

CONFUSION MATRICES

CLASSIFICATION ACCURACY VALUE

AREA UNDER THE ROC CURVE

% Use this section to present the methodologies that will generate results by
% analyzing the data. Suppose that the radius of a circle is $r$. Then its area is
% \begin{equation}
  % \label{eq:area}
  % \pi r^2.
% \end{equation}

% Equation~\eqref{eq:area} is interesting.

% Sometimes I don't want an equation to be numbered such as this one:
% \[
  % f(x) = \frac{1}{\sqrt{2\pi}} \exp\left( % - \frac{x^2}{2} \right),
% \]
% which is the density of a standard normal variable.

\section{Results}
\label{sec:resu}

OVERVIEW PARAGRAPH

OPTIONAL: PRESENT CONFUSION MATRICES FOR EACH OF THE METHODS

PRESENT ACC VALUES FOR EACH OF THE METHODS IN COMPARATIVE CHART

PRESENT ROC CURVES FOR EACH OF THE METHODS

PRESENT AUC VALUES FOR EACH OF THE METHODS IN COMPARATIVE CHART

% Table~\ref{tab:rv} summarizes some example draws from some distributions.

% \begin{table}[tbp]
 % \caption{This is my first table.}
 % \label{tab:rv}
% \centering
% \begin{tabular}{rrr}
 % \toprule
% normal & poisson & gamma \\ 
 % \midrule
% -0.110 & 4 & 2.401 \\ 
 % 0.116 & 4 & 3.529 \\ 
 % -0.828 & 9 & 2.112 \\ 
 % -0.066 & 6 & 11.104 \\ 
 % 0.219 & 3 & 4.815 \\ 
 % 0.303 & 5 & 2.188 \\ 
 % 0.544 & 0 & 8.050 \\ 
 % -2.617 & 8 & 3.646 \\ 
 % 0.747 & 1 & 5.178 \\ 
 % -1.103 & 4 & 3.043 \\ 
  % \bottomrule
% \end{tabular}
% \end{table}

% Figure~\ref{fig:cars} shows the distance against the speed from this dataset.


% \begin{figure}[tbp]
 % \centering
 % \includegraphics[width=\textwidth]{cars.pdf}
 % \caption{This is my first figure.}
 % \label{fig:cars}
% \end{figure}

\section{Discussion}
\label{sec:disc}

% What are the main contributions again?

% What are the limitations of this study?

% What are worth pursuing further in the future?

% Watch for prevalence of diabetes \citep{wild2004global}.

\bibliography{references}
\bibliographystyle{chicago}

\end{document}