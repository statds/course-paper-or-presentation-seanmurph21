\documentclass[12pt]{article}

%% preamble
\usepackage{amsmath}
\usepackage[margin = 1in]{geometry}
\usepackage{graphicx}
\usepackage{booktabs}
\usepackage{natbib}

% highlighting hyper links
\usepackage[colorlinks=true, citecolor=blue]{hyperref}


\title{Proposal: March Madness Brackets}
\author{Sean Murphy\\
  Statistics Major\\
  University of Connecticut
}

\begin{document}
\maketitle


\paragraph{Introduction}

Every year, Division One college basketball programs from across the country 
participate in a national tournament to determine the best program.  This  
tournament, which is immensely popular and takes place during March, has 
come to be known as "March Madness".  March Madness gets its name from various 
sources.  On the one hand, it is widely known for its upsets.  Teams heavily 
favored in match-ups lose with heightened regularity, and there are often 
"Cinderella" teams who advance further than anyone was expecting.  
Additionally, millions of Americans fill out brackets each year to predict 
the winner, and there is a tremendous amount of betting associated with 
these brackets.  All of this means that there is a pronounced interest in 
being able to predict outcomes as accurately as possible, making March 
Madness a fascinating topic for statistical modeling.  

Many statistical models of varying degrees of complexity have been developed 
to aid in tournament outcome prediction.  Models typically draw upon a variety 
of different factors to predict success.  Historical performance data, current 
season performance, rankings, and team composition metrics are some of the 
most commonly used.~\citep{toutkoushian2011predicting}.  Ranking systems 
themselves are often immensely statistically rigorous, and there are plenty of 
different systems in popular use today ~\citep{steinberg2018march}.  Moreover, 
models themselves have been constructed via various difference machine learning 
methods, such as linear regression, logistic regression, decision trees, 
nearest neighbors, random forest, and many others ~\citep{fonseca2018march}.

In this paper, I will build my own small-scale model to predict tournament game 
outcomes.  I will seek to identify, using updated data (2008-2022), which 
factors are the most important in predicting a team's success in tournament games.

\paragraph{Specific Aims}

The specific aim of this analysis is to determine which factors are the most 
significant in predicting March Madness tournament success.  This question can 
be broken down along a few dimensions.  First, it is useful to know which (if any) 
team statistics play a major role in predicting wins.  These team statistics can 
be defensive statistics such as points allowed, opponent's shooting percentages, 
turnovers caused, and defensive ratings.  They can also be offensive statistics 
such as shooting percentage, points scored, assists, offensive efficiency, etc.  

Additionally, I would like to investigate the ratings metrics developed by Ken 
Pomeroy and Bart Torvik to see how well they predict tournament success.  Since 
both have metrics covering adjusted offensive/defensive ratings, efficiency, and 
tempo, it would be useful to do a comparative analysis between these two popular 
metrics systems.  

\paragraph{Data}

The data that I will use for this analysis has been pulled from https://kenpom.com/ 
and https://www.barttorvik.com/ via Kaggle.  It includes updated data from 2008-2022 
on various facets of the NCAA tournament, including team data, trend data, game data, 
upset data, conference data, and more.  The data from the year 2020 are missing due 
to the tournament being cancelled during the COVID-19 pandemic.  Additionally, it 
includes designed metrics from both Ken Pomeroy and Bart Torvik on adjusted efficiency, 
offense, defense, and tempo, as specified above.  I will use the fields from these 
data sets to build the model.  

\paragraph{Research Design and Methods}

There are numerous ways to measure the success of a team in the NCAA tournament, but 
for this paper we shall simply treat the dependent variable as a two-level categorical 
variable (wins vs. losses).  Thus, since the predicted variable is categorical, this 
can be treated as a classification problem, which lends itself well to a logistic 
regression approach.  With a logistic regression model fitted, we will be able to see 
the probability associated with winning or losing for each team in a given tournament 
year ~\citep{fonseca2018march}. 

I will begin by examining the data, and selecting the factors that are often cited as 
most influential on game outcomes.  Then, I will fit a model including all these 
factors, aiming to identify those with significant influence on game outcomes.  I 
will eliminate predictors that are not identified as significant by the initial fit, 
and will create a new model with fewer predictors.  By this backward selection process 
I will seek to identify an optimal model for prediction, which can be evaluated by 
applying it to tournaments from previous years.  


\paragraph{Discussion}
Perhaps the most challenging aspect of this analysis lies in the potential for 
collinearity among the predictors.  For instance, it is likely that there may be strong 
collinearity present among offensive or defensive statistical predictors, which will 
have to be sorted out in the model construction process.  Teams that perform quite well 
in one of these areas often tend to do well in their respective subcategories.

One limitation of this analysis is that I will not be addressing a pressing question 
that often arises with respect to the NCAA tournament.  Namely, I will not attempt to 
identify the factors that most accurately predict upsets in the course of the tournament.  
Given the time and scope of this project, this simply does not seem feasible in addition 
to the intended analysis described above.  

As a fall back plan, I can always identify different data sources that contain more 
information related to the model I am building.  Additionally, I can implement various 
different types of models and compare their effectiveness. 

\bibliography{proposalreferences}
\bibliographystyle{chicago}

\end{document}