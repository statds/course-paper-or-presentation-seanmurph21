\documentclass[12pt]{article}

%% preamble
\usepackage{amsmath}
\usepackage[margin = 1in]{geometry}
\usepackage{graphicx}
\usepackage{booktabs}
\usepackage{natbib}

% highlighting hyper links
\usepackage[colorlinks=true, citecolor=blue]{hyperref}


\title{Proposal: March Madness Brackets}
\author{Sean Murphy\\
  Statistics Major\\
  University of Connecticut
}

\begin{document}
\maketitle


\paragraph{Introduction}

Why is the topic interesting?
Every year, Division One college basketball programs from across the country participate in a national tournament to determine the best program.  This tournament, which is immensely popular and takes place during March, has come to be known as "March Madness".  March Madness gets its name from various sources.  On the one hand, it is widely known for its upsets.  Teams heavily favored in match-ups lose with heightened regularity, and there are often "Cinderella" teams who advance further than anyone was expecting.  Additionally, millions of Americans fill out brackets each year to predict the winner, and there is a tremendous amount of betting associated with these brackets.  All of this means that there is a pronounced interest in being able to predict outcomes as accurately as possible, making March Madness a fascinating topic for statistical modeling.  

What has been done?
Many statistical models of varying degrees of complexity have been developed to aid in tournament outcome prediction.  Models typically draw upon a variety of different factors to predict success.  Historical performance data, current season performance, rankings, and team composition metrics are some of the most commonly used.~\citep{toutkoushian2011predicting}.  Ranking systems themselves are often immensely statistically rigorous, and there are plenty of different systems in popular use today ~\citep{steinberg2018march}.  Moreover, models themselves have been constructed via various difference machine learning methods, such as linear regression, logistic regression, decision trees, nearest neighbors, random forest, and many others ~\citep{}

What is new?

Background about your research; touch the three questions to be addressed in an
introduction; cite relevant references~\citep[e.g.,][]{dwivedi2017analysis}.


\paragraph{Specific Aims}

Overall Research Question:

Specific Statistical Questions that can be investigated

Formulate your research question;
translate your research question into statistical/data science questions

\paragraph{Data}

Introduce the data:
Source
Summary

Hopefully, you have identified the data needed for your project. Give a
description about it.

\paragraph{Research Design and Methods}

Discuss Methodology for how you will answer the question

What design or methods will you use?
Cite relevant references~\citep[e.g.,][]{wild2004global}.

\paragraph{Discussion}
What are the most challenge parts of the task?

What are the limitations of your work?

What is your fall-back plan if
something unexpected happens?

\bibliography{proposalreferences}
\bibliographystyle{chicago}

\end{document}